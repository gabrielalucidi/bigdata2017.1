\documentclass[numbers,a4paper,12pt]{article}
\usepackage[brazil]{babel}
\usepackage{natbib}
\usepackage{url}
%\usepackage[latin1]{inputenc}
\usepackage[utf8x]{inputenc}
\usepackage{amsmath,amssymb}
\usepackage{graphicx}
\graphicspath{{images/}}
\usepackage{parskip}
\usepackage{fancyhdr}
\usepackage{algorithmic}
\usepackage{algorithm}
\usepackage{float}
\usepackage{color}
\usepackage{epstopdf}
\usepackage{epsfig}
\usepackage{pdfpages}
\usepackage{multirow}
\usepackage[export]{adjustbox}
\usepackage{vmargin}
\usepackage{textcomp}
\setmarginsrb{3 cm}{2.5 cm}{3 cm}{2.5 cm}{1 cm}{1.5 cm}{1 cm}{1.5 cm}

\newtheorem{definition}{Definição}
\newtheorem{example}{Exemplo}
\newtheorem{algoritmo}{Algoritmo}
\newtheorem{theorem}{Teorema}
\newtheorem{remark}{Observação}

\newcommand{\vecx}{\underline{x}}

\title{Proposta de projeto}	% Title
\author{Big Data}			% Author
\date{\today}				% Date

\makeatletter
\renewcommand*\env@matrix[1][*\c@MaxMatrixCols c]{%
	\hskip -\arraycolsep
	\let\@ifnextchar\new@ifnextchar
	\array{#1}}
\makeatother
\makeatletter
\let\thetitle\@title
\let\theauthor\@author
\let\thedate\@date
\makeatother

\pagestyle{fancy}
\fancyhf{}
\rhead{\theauthor}
\lhead{\thetitle}
\cfoot{\thepage}

\begin{document}

%%%%%%%%%%%%%%%%%%%%%%%%%%%%%%%%%%%%%%%%%%%%%%%%%%%%%%%%%%%%%%%%%%%%%%%%%%%%%%%%%%%%%%%%%

\begin{titlepage}
	\centering
    \vspace*{0.5 cm}
    \includegraphics[scale = 0.3]{logo.jpg}\\[1.0 cm]	% University Logo
    \textsc{\LARGE Universidade Federal do Rio de Janeiro\newline\newline}\\[2.0 cm]	% University Name
	\textsc{ Engenharia de Controle e Automação}\\[0.5 cm]				% Course Code
	\rule{\linewidth}{0.2 mm} \\[0.4 cm]
	{ \huge \bfseries \thetitle}\\
	\rule{\linewidth}{0.2 mm} \\[1.5 cm]
	
	\begin{minipage}{0.4\textwidth}
		\begin{flushleft} \large
			\emph{Big Data}\\
			Professor : \\Alexandre de Assis\\
			\end{flushleft}
			\end{minipage}~
			\begin{minipage}{0.4\textwidth}
            
			\begin{flushright} \large
			\emph{Aluno(s) :} \\
			Gabriel Pelielo\\
            Gabriel Premoli Monteiro\\
            Gabriela Lúcide\\
            Jean Américo\\
		\end{flushright}
       	

   
	\end{minipage}\\[2 cm]
	
\begin{center}
	\small \today
\end{center}
    
    
    
    
	
\end{titlepage}

%%%%%%%%%%%%%%%%%%%%%%%%%%%%%%%%%%%%%%%%%%%%%%%%%%%%%%%%%%%%%%%%%%%%%%%%%%%%%%%%%%%%%%%%%

\tableofcontents
\pagebreak

%%%%%%%%%%%%%%%%%%%%%%%%%%%%%%%%%%%%%%%%%%%%%%%%%%%%%%%%%%%%%%%%%%%%%%%%%%%%%%%%%%%%%%%%%

\section{Objetivo}

%\begin{figure}[H]
%	\begin{center}
%		\includegraphics[]{}   
%		%\vspace{-0.15cm}
%		\caption{}
%		\label{}
%	\end{center}
%\end{figure}}

A disciplina de Big Data oferece a oportunidade dos alunos implementarem o conteúdo oferecido por meio de um projeto. Com esse objetivo, levantamos a seguinte proposta: 

Desenvolver uma aplicação que utiliza o Apache Spark para o processamento de dados obtidos por meio de uma API externa do facebook (em formato .JSON) relativos à eventos públicos, utilizando o resultado de seu processamento para produzir inicialmente um mapa de calor condizente à concentração de eventos numa dada área.

Nesse sentido, a restrição de Big Data estaria intimamente ligada ao volume de dados retornado pela API.

\section{Requisitos funcionais}

Como proposta inicial, os requisitos funcionais objetivados são: 

\begin{itemize}
\item Permitir a pesquisa de eventos públicos do facebook a partir de uma interface de busca;
\item A interface de busca deve permitir que o usuário defina critérios de pesquisa;
\item A princípio, o critério de pesquisa a ser definido deve ser apenas um: localização de eventos, sendo este uma lista de bairros cariocas;
\item A interface de busca deve retornar seus resultados em forma de um mapa de calor;
\item O mapa de calor deve evidenciar os pontos do bairro carioca escolhido onde se concentram os eventos públicos do facebook.
\end{itemize}

Espera-se que, implementadas tais funções, novas funcionalidades sejam consideradas.

\section{Composição do grupo}

O grupo é composto de quatro componentes. São eles:

\begin{itemize}
	\item Gabriel Pelielo - gabrielpelielo@poli.ufrj,br ;
	\item Gabriel Premoli - gabriel\_premolimonteiro@poli.ufrj.br ;
	\item Gabriela Lúcide - gpinhao@poli.ufrj.br ;
	\item Jean Américo - jamerico@poli.ufrj.br ;
\end{itemize}

\section{Ferramentas a serem utilizadas}

Pretende-se utilizar:

\begin{enumerate}
	\item GitHub;
	\item Apache Spark;
	\item API facebook;
	\item XAMPP;
	\item Hadoop File Sistem (HDFS);
	\item Google charts;
\end{enumerate}

%\newpage

%\nocite{*}
%\bibliographystyle{IEEEtran}
%\bibliography{biblist}

%Credits -Divjot Kaur - Linus
%Based on "Operating Systems Laboratory"
%https://www.overleaf.com/latex/templates/operating-systems-laboratory/wswyyqztyryp#.WPJmndIrJhE

\end{document}